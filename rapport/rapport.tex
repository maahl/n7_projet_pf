\documentclass[a4paper]{article}

% Options possibles : 10pt, 11pt, 12pt (taille de la fonte)
%                     oneside, twoside (recto simple, recto-verso)
%                     draft, final (stade de développement)

\usepackage[utf8]{inputenc}   % LaTeX, comprends les accents !
\usepackage[T1]{fontenc}      % Police contenant les caracteres français
\usepackage[francais]{babel}  % Placez ici une liste de langues, la
                              % derniere etant la langue principale

\usepackage[a4paper]{geometry}% Reduire les marges
% \pagestyle{headings}        % Pour mettre des entetes avec les titres
                              % des sections en haut de page

\title{Interpréteur LOGO en OCaml}           % Les parametres du titre : titre, auteur, date
\author{Maxence \bsc{Ahlouche}}
\date{}                       % La date n'est pas requise (la date du
                              % jour de compilation est utilisee en son
			      % absence)

\sloppy                       % Ne pas faire deborder les lignes dans la marge

\begin{document}

\maketitle                    % Faire un titre utilisant les donnees
                              % passees a  \title, \author et \date

%\begin{abstract}
%\end{abstract}

\tableofcontents              % Table des matieres

\part{Présentation du projet}                % Commencer une partie...

\section{Donec}               % Commencer une section, etc.


\subsection{Praesent}         % Section plus petite

% \subsubsection{Titre}       % Encore plus petite

\subsection{Quisque}

% \paragraph{Titre}           % Toutes petites sections (le nom \paragraph
                              % n'est pas tres bien choisi)

% \subparagraph{Titre}        % La derniere

% \appendix                   % Commencons les annexes

% \section{Titre}             % Annexe A

% \section{Titre}             % Annexe B

% \listoffigures              % Table des figures

% \listoftables               % Liste des tableaux

\end{document}


